\documentclass[a4paper,12pt]{article}

%%% Работа с русским языком
\usepackage{cmap}					% поиск в PDF
\usepackage{mathtext} 				% русские буквы в формулах
\usepackage[T2A]{fontenc}			% кодировка
\usepackage[utf8]{inputenc}			% кодировка исходного текста
\usepackage[english,russian]{babel}	% локализация и переносы

%%% Страница
\usepackage{extsizes} % Возможность сделать 14-й шрифт
\usepackage{geometry} % Создание полей
\geometry{top = 2cm}
\geometry{bottom = 2cm}
\geometry{left = 2.5cm}
\geometry{right = 2.5cm}
\renewcommand{\baselinestretch}{1.5} % Интерлиньяж 1.5

%%% Дополнительная работа с математикой
\usepackage{amsmath,amsfonts,amssymb,amsthm,mathtools} % AMS
\usepackage{icomma} % "Умная" запятая: $0,2$ --- число, $0, 2$ --- перечисление

%% Номера формул
%\mathtoolsset{showonlyrefs=true} % Показывать номера только у тех формул, на которые есть \eqref{} в тексте.
%\usepackage{leqno} % Нумерация формул слева

%% Свои команды
\DeclareMathOperator{\sgn}{\mathop{sgn}}

%% Перенос знаков в формулах (по Львовскому)
\newcommand*{\hm}[1]{#1\nobreak\discretionary{}
	{\hbox{$\mathsurround=0pt #1$}}{}}

%%% Работа с картинками
\usepackage{graphicx}  % Для вставки рисунков
\graphicspath{{images/}{images2/}}  % папки с картинками
\setlength\fboxsep{3pt} % Отступ рамки \fbox{} от рисунка
\setlength\fboxrule{1pt} % Толщина линий рамки \fbox{}
\usepackage{wrapfig} % Обтекание рисунков текстом

%%% Работа с таблицами
\usepackage{array,tabularx,tabulary,booktabs} % Дополнительная работа с таблицами
\usepackage{longtable}  % Длинные таблицы
\usepackage{multirow} % Слияние строк в таблице

%%% Теоремы
\theoremstyle{plain} % Стиль по умолчанию
\newtheorem{theorem}{Теорема}[section]
\newtheorem{proposition}[theorem]{Утверждение}

\theoremstyle{definition} % "Определение"
\newtheorem{corollary}{Следствие}[theorem]
\newtheorem{problem}{Задача}[section]

\theoremstyle{remark} % "Примечание"
\newtheorem*{nonum}{Решение}

%%% Программирование
\usepackage{etoolbox} % логические операторы

\usepackage{lastpage} % Узнать, сколько всего страниц в документе.

\usepackage{soul} % Модификаторы начертания


\begin{document}
	
	\begin{center}
		
		{\Large\textbf{Обзор}}
		
		\textbf{Название проекта: }
		
		"Изучение навыков работы в программе SALOME. Изучение работы с командами консоли. Написание скриптов для SALOME на языке Python."
		
		\textit{Авторы: Есис А. И., Павленок Л. Л.}
		
		\textit{Руководитель проекта: Чмыхов М. А.}
		
	\end{center}
	
	3D-моделирование стало неотъемлемой частью нашей жизни. Сегодня оно широко используется в сфере маркетинга, архитектурного дизайна и кинематографии, не говоря уже о промышленности. 3D-моделирование позволяет cоздать прототип будущего сооружения, коммерческого продукта в объемном формате.
	
	Cегодня невозможно представить 2 крупные отрасли без применения трехмерных моделей. Это — промышленность и индустрия развлечений.
	С индустрией развлечений мы сталкиваемся почти каждый день. Это фильмы, анимация и компьютерные игры. С помощью одного и того же принципа — полигонального моделирования cозданы все виртуальные миры и персонажи. Полигональное моделирование происходит путем манипуляций с полигонами в пространстве. Вытягивание, перемещение вращение,  и.т.д. Однако невозможно контролировать необходимые зазоры, сечения, учесть физические свойства материала и технологию изготовления, используя полигональное моделирование. Для таких изделий используются методы промышленного проектирования.
	
	CAПР (Система Aвтоматизированного Поектирования). Современные CAD-подсистемы, входящие в состав интегрированных САD/CAM/CAE-систем — поддержанное компьютером конструирование/изготовление/инженерная деятельность), и системы твердотельного параметрического моделирования механических объектов, отражающие последние достижения инженерной компьютерной графики, представляют собой наиболее важные разработки в области новых технологий по автоматизации деятельности инженеров,	конструкторов и технологов. По уровню возможностей САПР условно разделяются на три уровня:
	\begin{enumerate}
		\item САПР нижнего уровня — это, чаще всего, программы для двумерного проектирования.
		\item САПР среднего уровня — позволяют дополнительно создавать трехмерные параметрические модели и выполнять проверочные расчеты деталей и сборок.
		\item САПР верхнего уровня — обеспечивают потребности практически всех областей проектирования, от разработки изделий и оснастки до проведения сложных инженерных расчетов и изготовления изделий\cite{bBolshakov}.
	\end{enumerate}
	В данной работе мы используем САПР среднего уровня, которой является Salome-MECA. Изначально задуманная в качестве связующего ПО CAD-CAE, она включает в себя разнообразные модули, используемые в приложениях численного моделирования — от моделирования в САПР до параллельных вычислений. Одним из важнейших преимуществ Salome-MECA является то, что её исходный код открыт и бесплатно распространяется на условиях GNU Lesser General Public License. Это означает, что Salome может расширяться за счет сторонних свободных или коммерческих модулей. Внутренним языком платформы является Python, причем в самой платформе имеется встроенная консоль, которая может быть использованна для выполнения пользовательских сценариев и автоматизации обработки множества типовых задач (пакетной обработки)\cite{wOfDoc}. Платформа Salome, как и любая CAE-система, состоит из четырех модулей: решателя (Solver), поcтроителя сетки (Mesh), модуля поcтроения геoметрии (Geometry), модуля oбработки и представления результатов (Pоst-processing).
	
	Метод конечных элементов лежит в основе большинства CAE-пакетов . Идея этого метода заключается в замене непрерывной функции, описывающей изучаемое явление или процесс, дискретной моделью, которая строится на базе множества кусочно-непрерывных функций, определенных на конечном числе подобластей. Каждая такая подобласть конечна и представляет собой часть (элемент) всей области, поэтому их называют конечными элементами. Иcследуемая геометрическая область разбивается на элементы так, чтобы на каждом из них неизвестная функция аппроксимировалась пробной функцией. Такое разбиение называется расчетной сеткой. Создание хорошей расчетной сетки также представляет собой нетривиальную задачу. Это связано с тем, что реальные детали машин имеют сложную геометрию и необходимо разделить их на такие элементы, чтобы приближенные решения не сильно отличались от точных. Поэтому, кроме самих CAE-пакетов, существует бoльшое число приложений, выполняющих всего одну важную функцию: построение расчетной сетки \cite{wIbmSalome}. В данной работе наибольший интерес представляют модуль Geometry и Mesh, поскольку для них предусмотренно написание скриптов.
	
	Любая задача моделирования начинается с создания геометрической модели изучаемого объекта – определения его формы. Другими cловами, необходимо задать границы той части пространства, поведение которой необходимо исследовать. Геометрическая модель может представлять собой двумерный или трехмерный чертеж.
	
	Геометрический модуль SALOME разработан для:
	\begin{itemize}
		\item Построения и оптимизации геометрических моделей, используя широкий набор функций автоматизированного проектирования
		\item Создание элементарных геометрических объектов
		\item Поcтроение примитивов
		\item Построение форм
		\item Генерирование сложных форм
		\item Работа c группами
		\item Геометрические булевы (логические) операции
		\item Геометрические преобразования
		\item Построение блоками
	\end{itemize}
	
	В геометрический модуль можно как экспортировать, так и импортировать геометрические объекты следующих форматов: BREP, IGES, STEP. Механизмы импорта и экспорта осуществлены через дополнения (плагины), что дает возможность расширить диапазон доступных форматов, добавляя другие приложения-дополнения (плагины) (такие как CATIA 5)\cite{wLadugaGeom}.
	
	Любой реальный исследуемый объект чаще всего обладает достаточно сложной формой. И описать его поведение при наложении внешних нагрузок, как цельное, достаточно сложно хотя бы потому, что любой объект чаще всего имеет в своем составе хотя бы несколько элементов. Следовательно, нужно записать точные математические выражения для граничных условий на поверхности каждого типа, а затем совместить их.
	
	Можно использовать другой алгоритм, разбивая всю деталь на очень мелкие фрагменты определенной формы (конечные элементы): треугольники, четырехугольники, шестиугольники и т.д. для 2D-объектов и тетраэдры, гексаэдры и т.д. для 3D-объектов. Граничные условия для каждого такого фрагмента будут иметь достаточно простой вид. Более того, из-за малых размеров элемента, решение для него можно аппроксимировать при помощи элементарных функций, например, линейным или квадратичным полиномом. Используя данный алгоритм, мы получим систему алгебраических уравнений, для решения которой существует большое количество легко реализуемых методов.
	
	При разбиении объекта на малые элементы сначала необходимо определить точки, которые будут вершинами элементов. Эти точки называют узлами (node) расчетной сетки или 0D-элементами. Далее, на основе выбранных алгоритмов и геометрии базовых элементов, узлы соединяются прямыми линями, называемыми ребрами (edge) или 1D-элементами. Область, заключенная между несколькими ребрами и не содержащая ни одного узла или ребра, называется гранью (face) или 2D-элементом. Обычно грани образуются тремя или четырьмя ребрами. Область, заключенная между несколькими гранями и не содержащая ни одного узла, ребра или грани (части грани) называется объемным (volume object) или 3D-элементом.
	
	Разбиение объекта на элементы в Salome выполняется по выбираемым пользователем алгоритмам и на основе параметров, оформляемых в виде гипотез \cite{wIbmSalomeMesh}.
	
	Модуль MESH SALOME предназначен для\cite{wOfDoc}:
	\begin{itemize}
		\item Импорта и экспорта сеток в формате MED
		\item Сцепление геометрических моделей, ранее созданных или импортированных компонентом Geometry
		\item Модификации в местном масштабе сгенерированных сеток
		\item Суммирования узлов и элементов (Addition of nodes and elements)
		\item Удаления узлов и элементов (Removal of nodes and elements)
		\item Контроля качества сеток, базирующихся на ряде определенных критериев
	\end{itemize}
	
	\newpage
	\bibliographystyle{utf8gost705u}  %% стилевой файл для оформления по ГОСТу
	\bibliography{Obzor}     %% имя библиографической базы (bib-файла) 
	
\end{document}